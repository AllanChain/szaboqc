\chapter{双电子的自洽场程序}

这个附录给出了一个实现从头算Hartree-Fock方法的小程序的FORTRAN代码和实例输出($\mathrm{HeH}^+$)。
$\mathrm{HeH}^+$的计算在\autoref{sec3.5.3}已经讨论过了。

这个程序可以计算STO-NG(N=1,2或3)波函数下任意的双电子双原子分子。在主程序HFCALC中的输出参数是
输出的控制选项;一个$1s$ Slater函数展开需要的原初$1s$高斯函数的的数目,例如STO-NG中的N;
原子单位下的键长$R$;$1s$ Slater函数的两个指数$\zeta_1$和$\zeta_2$以及两个原子核的原子数$Z_A$和$Z_B$。
如果
\begin{align}
	\label{B.1}
	g_{1s}(\alpha)=(2\alpha/\pi)^{3/4}e^{-\alpha r^2}
\end{align}
是一个归一化的原初$1s$高斯函数,然后这个程序可以使用每一个都经过下面三个最小二乘法拟合的Slater型函数作为基函数
\begin{align}
	\label{B.2}
	\psi_{1s}^{CGF}(\zeta=1.0,STO-1G)=g_{1s}(0.270950)
\end{align}
\begin{align}
	\label{B.3}
	\psi_{1s}^{CGF}(\zeta=1.0,STO-2G)=0.678914g_{1s}(0.151623)+0.430129g_{1s}(0.851819)
\end{align}
\begin{align}
	\label{B.4}
	\psi_{1s}^{CGF}(\zeta=1.0,STO-3G)=&0.444635g_{1s}(0.109818)+0.535328g_{1s}(0.405771)
    \nonumber\\
                                      &+0.154329g_{1s}(2.22766)
\end{align}
这个程序最近最多运行在装有FORTRAN IV 编译器的一台PDP-10计算机上。但是,它应该在几乎任何FORTRAN编译器上都可以正确编译运行。
在IBM的机器上DERF函数可以被一个标准库例程代替。这个程序使用双精度变量,但是对于许多情况来说,单精度也足够满足要求。
这个程序特意写成了一种不是很效率的形式,以此为了使任何懂FORTRAN和了解\autoref{ch:3}讨论的人可以完全的理解从头算Hartree-Fock方法
的所有细节。

子程序INTGRL在标度完给定$\zeta_1$和$\zeta_2$值下的由\autoref{B.2}到\autoref{B.4}收缩系数和指数后,使用\autoref{appendix:a}中
给出的精确方程计算了所有基本的单电子和双电子积分。未归一化的原初重叠积分、动能积分、核吸引势积分和双电子积分由函数S,T,V和TWOE计算,
然后用包含归一化常数的收缩系数相乘并求和。随后这些基本的积分从COMMON转存到COLLECT,这些积分被用来构造$\bf{S}$、$\bf{H^{core}}$等
矩阵,并在自洽场迭代的过程中保持不变。正交基函数的转换矩阵$\bf{X}$如文中讨论的一样是一个酉矩阵。这个自洽场迭代流程使用芯哈密顿量作为
Fock矩阵的初猜,收敛条件被定为密度矩阵元素的标准差小于$10^{-4}a.u.$。最大的迭代轮数为25。除了能量以外没有计算其他的期望值,但是矩阵
$\bf{PS}$被输出出来进行Mulliken布居分析。

这个程序很简单,但是它包含了进行大型从头算的复杂程序的基本的组成部分。积分程序S,T,V和TWOE可以广泛的用于任何基组中$1s$的高斯函数。
对于有兴趣和能力并且完全理解了现在这个程序的读者,可以将这个程序扩展为可以使用不确定的$1s$高斯函数基组进行多原子模型计算,比如使用
Whitten的“Gaussian lobe”基组。\endnote{J.L.Whitten,Gaussian lobe function expansions of Hartree-Fock solutions for the 
first row atoms and ethtlene,\textit{J.Chem.Phys. }\textbf{44:359(1966).}}计算如果使用超过两个基函数的时候,需要用到更加一般化的
矩阵对角化程序\endnote{A listing of an efficient diagonalization program is given by J.A.Pople and D.L.Beveridge,
\textit{Approximate Molecular Orbital Theory, }McGraw-Hill,New York ,1970,Appendix 2.},
而不是像我们之前那样只考虑$2\times2$的情况。许多多原子的程序
除了上述提到的以外,还要求一些专有的存储和操作数目庞大的双电子积分的技巧以及一个高效的从双电子积分和密度矩阵构建矩阵$\bf{G}$的算法。
计算p型和d型笛卡尔高斯函数的积分被认为比计算$1s$的更加困难,并且是编写一个高效的多原子计算程序的一项重点工作。
\endnote{J.S.Binkley,R.A.Whiteside,R.Krushnan,R.Seeger,H.B.Schlegel,D.J.Defrees,and J.A.Pople,
Gaussian 80,program \#406,\textit{Quantum Chemistry Program Exchange},\textbf{Indiana University.}}
\newpage
% http://www.ccl.net/cca/software/SOURCES/FORTRAN/szabo/index.html
\begin{verbatim}
C********************************************************************
C
C     MINIMAL BASIS STO-3G CALCULATION ON HEH+
C
C     THIS IS A LITTLE DUMMY MAIN PROGRAM WHICH CALLS HFCALC
C
C*********************************************************************
      IMPLICIT DOUBLE PRECISION(A-H,O-Z)
      IOP=2
      N=3
      R=1.4632D0
      ZETA1=2.0925D0
      ZETA2=1.24D0
      ZA=2.0D0
      ZB=1.0D0
      CALL HFCALC(IOP,N,R,ZETA1,ZETA2,ZA,ZB)
      END
C*********************************************************************
      SUBROUTINE HFCALC(IOP,N,R,ZETA1,ZETA2,ZA,ZB)
C
C DOES A HARTREE-FOCK CALCULATION FOR A TWO-ELECTRON DIATOMIC
C USING THE 1S MINIMAL STO-NG BASIS SET
C MINIMAL BASIS SET HAS BASIS FUNCTIONS 1 AND 2 ON NUCLEI A AND B
C
C IOP=0 NO PRINTING WHATSOEVER (TO OPTIMIZE EXPONENTS, SAY)
C IOP=1 PRINT ONLY CONVERGED RESULTS
C IOP=2 PRINT EVERY ITERATION
C N     STO-NG CALCULATION (N=1,2 OR 3)
C R     BONDLENGTH (AU)
C ZETA1 SLATER ORBITAL EXPONENT (FUNCTION 1)
C ZETA2 SLATER ORBITAL EXPONENT (FUNCTION 2)
C ZA    ATOMIC NUMBER (ATOM A)
C ZB    ATOMIC NUMBER (ATOM B)
C
C*********************************************************************
      IMPLICIT DOUBLE PRECISION(A-H,O-Z)
      IF (IOP.EQ.0) GO TO 20
      PRINT 10,N,ZA,ZB
   10 FORMAT(' ',2X,'STO-',I1,'G FOR ATOMIC NUMBERS ',F5.2,' AND ',
     $ F5.2//)
   20 CONTINUE
C CALCULATE ALL THE ONE AND TWO ELECTRON INTEGRALS
      CALL INTGRL(IOP,N,R,ZETA1,ZETA2,ZA,ZB)
C BE INEFFICIENT AND PUT ALL INTEGRALS IN PRETTY ARRAYS
      CALL COLECT(IOP,N,R,ZETA1,ZETA2,ZA,ZB)
C PERFORM THE SCF CALCULATION
      CALL SCF(IOP,N,R,ZETA1,ZETA2,ZA,ZB)
      RETURN
      END
C*********************************************************************
      SUBROUTINE INTGRL(IOP,N,R,ZETA1,ZETA2,ZA,ZB)
C
C     CALCULATES ALL THE BASIC INTEGRALS NEEDED FOR SCF CALCULATION
C
C*********************************************************************
      IMPLICIT DOUBLE PRECISION(A-H,O-Z)
      COMMON/INT/S12,T11,T12,T22,V11A,V12A,V22A,V11B,V12B,V22B,
     $ V1111,V2111,V2121,V2211,V2221,V2222
      DIMENSION COEF(3,3),EXPON(3,3),D1(3),A1(3),D2(3),A2(3)
      DATA PI/3.1415926535898D0/
C THESE ARE THE CONTRACTION COEFFICIENTS AND EXPONENTS FOR
C A NORMALIZED SLATER ORBITAL WITH EXPONENT 1.0 IN TERMS OF
C NORMALIZED 1S PRIMITIVE GAUSSIANS
      DATA COEF,EXPON/1.0D0,2*0.0D0,0.678914D0,0.430129D0,0.0D0,
     $ 0.444635D0,0.535328D0,0.154329D0,0.270950D0,2*0.0D0,0.151623D0,
     $ 0.851819D0,0.0D0,0.109818D0,0.405771D0,2.22766D0/
      R2=R*R
C SCALE THE EXPONENTS (A) OF PRIMITIVE GAUSSIANS
C INCLUDE NORMALIZATION IN CONTRACTION COEFFICIENTS (D)
      DO 10 I=1,N
      A1(I)=EXPON(I,N)*(ZETA1**2)
      D1(I)=COEF(I,N)*((2.0D0*A1(I)/PI)**0.75D0)
      A2(I)=EXPON(I,N)*(ZETA2**2)
      D2(I)=COEF(I,N)*((2.0D0*A2(I)/PI)**0.75D0)
   10 CONTINUE
C D AND A ARE NOW THE CONTRACTION COEFFICIENTS AND EXPONENTS
C IN TERMS OF UNNORMALIZED PRIMITIVE GAUSSIANS
      S12=0.0D0
      T11=0.0D0
      T12=0.0D0
      T22=0.0D0
      V11A=0.0D0
      V12A=0.0D0
      V22A=0.0D0
      V11B=0.0D0
      V12B=0.0D0
      V22B=0.0D0
      V1111=0.0D0
      V2111=0.0D0
      V2121=0.0D0
      V2211=0.0D0
      V2221=0.0D0
      V2222=0.0D0
C CALCULATE ONE-ELECTRON INTEGRALS
C CENTER A IS FIRST ATOM, CETER B IS SECOND ATOM
C ORIGIN IS ON CENTER A
C V12A = OFF-DIAGONAL NUCLEAR ATTRACTION TO CENTER A, ETC.
      DO 20 I=1,N
      DO 20 J=1,N
C RAP2 = SQUARED DISTANCE BETWEEN CENTER A AND CENTER P, ETC.
      RAP=A2(J)*R/(A1(I)+A2(J))
      RAP2=RAP**2
      RBP2=(R-RAP)**2
      S12=S12+S(A1(I),A2(J),R2)*D1(I)*D2(J)
      T11=T11+T(A1(I),A1(J),0.0D0)*D1(I)*D1(J)
      T12=T12+T(A1(I),A2(J),R2)*D1(I)*D2(J)
      T22=T22+T(A2(I),A2(J),0.0D0)*D2(I)*D2(J)
      V11A=V11A+V(A1(I),A1(J),0.0D0,0.0D0,ZA)*D1(I)*D1(J)
      V12A=V12A+V(A1(I),A2(J),R2,RAP2,ZA)*D1(I)*D2(J)
      V22A=V22A+V(A2(I),A2(J),0.0D0,R2,ZA)*D2(I)*D2(J)
      V11B=V11B+V(A1(I),A1(J),0.0D0,R2,ZB)*D1(I)*D1(J)
      V12B=V12B+V(A1(I),A2(J),R2,RBP2,ZB)*D1(I)*D2(J)
      V22B=V22B+V(A2(I),A2(J),0.0D0,0.0D0,ZB)*D2(I)*D2(J)
   20 CONTINUE
C CALCULATE TWO-ELECTRON INTEGRALS
      DO 30 I=1,N
      DO 30 J=1,N
      DO 30 K=1,N
      DO 30 L=1,N
      RAP=A2(I)*R/(A2(I)+A1(J))
      RBP=R-RAP
      RAQ=A2(K)*R/(A2(K)+A1(L))
      RBQ=R-RAQ
      RPQ=RAP-RAQ
      RAP2=RAP*RAP
      RBP2=RBP*RBP
      RAQ2=RAQ*RAQ
      RBQ2=RBQ*RBQ
      RPQ2=RPQ*RPQ
      V1111=V1111+TWOE(A1(I),A1(J),A1(K),A1(L),0.0D0,0.0D0,0.0D0)
     $ *D1(I)*D1(J)*D1(K)*D1(L)
      V2111=V2111+TWOE(A2(I),A1(J),A1(K),A1(L),R2,0.0D0,RAP2)
     $ *D2(I)*D1(J)*D1(K)*D1(L)
      V2121=V2121+TWOE(A2(I),A1(J),A2(K),A1(L),R2,R2,RPQ2)
     $ *D2(I)*D1(J)*D2(K)*D1(L)
      V2211=V2211+TWOE(A2(I),A2(J),A1(K),A1(L),0.0D0,0.0D0,R2)
     $ *D2(I)*D2(J)*D1(K)*D1(L)
      V2221=V2221+TWOE(A2(I),A2(J),A2(K),A1(L),0.0D0,R2,RBQ2)
     $ *D2(I)*D2(J)*D2(K)*D1(L)
      V2222=V2222+TWOE(A2(I),A2(J),A2(K),A2(L),0.0D0,0.0D0,0.0D0)
     $ *D2(I)*D2(J)*D2(K)*D2(L)
   30 CONTINUE
      IF (IOP.EQ.0) GO TO 90
      PRINT 40
   40 FORMAT(3X,'R',10X,'ZETA1',6X,'ZETA2',6X,'S12',8X,'T11'/)
      PRINT 50, R,ZETA1,ZETA2,S12,T11
   50 FORMAT(5F11.6//)
      PRINT 60
   60 FORMAT(3X,'T12',8X,'T22',8X,'V11A',7X,'V12A',7X,'V22A'/)
      PRINT 50, T12,T22,V11A,V12A,V22A
      PRINT 70
   70 FORMAT(3X,4HV11B,7X,4HV12B,7X,4HV22B,7X,'V1111',6X,'V2111'/)
      PRINT 50, V11B,V12B,V22B,V1111,V2111
      PRINT 80
   80 FORMAT(3X,5HV2121,6X,5HV2211,6X,5HV2221,6X,5HV2222/)
      PRINT 50, V2121,V2211,V2221,V2222
   90 RETURN
      END
C*********************************************************************
      FUNCTION F0(ARG)
C
C     CALCULATES THE F FUNCTION
C     FO ONLY (S-TYPE ORBITALS)
C
C*********************************************************************
      IMPLICIT DOUBLE PRECISION(A-H,O-Z)
      DATA PI/3.1415926535898D0/
      IF (ARG.LT.1.0D-6) GO TO 10
C F0 IN TERMS OF THE ERROR FUNCTION
      F0=DSQRT(PI/ARG)*DERFOTHER(DSQRT(ARG))/2.0D0
      GO TO 20
C ASYMPTOTIC VALUE FOR SMALL ARGUMENTS
   10 F0=1.0D0-ARG/3.0D0
   20 CONTINUE
      RETURN
      END
C*********************************************************************
      FUNCTION DERFOTHER(ARG)
C
C     CALCULATES THE ERROR FUNCTION ACCORDING TO A RATIONAL
C     APPROXIMATION FROM M. ARBRAMOWITZ AND I.A. STEGUN,
C     HANDBOOK OF MATHEMATICAL FUNCTIONS, DOVER.
C     ABSOLUTE ERROR IS LESS THAN 1.5*10**(-7)
C     CAN BE REPLACED BY A BUILT-IN FUNCTION ON SOME MACHINES
C
C*********************************************************************
      IMPLICIT DOUBLE PRECISION(A-H,O-Z)
      DIMENSION A(5)
      DATA P/0.3275911D0/
      DATA A/0.254829592D0,-0.284496736D0,1.421413741D0,
     $ -1.453152027D0,1.061405429D0/
      T=1.0D0/(1.0D0+P*ARG)
      TN=T
      POLY=A(1)*TN
      DO 10 I=2,5
      TN=TN*T
      POLY=POLY+A(I)*TN
   10 CONTINUE
      DERFOTHER=1.0D0-POLY*DEXP(-ARG*ARG)
      RETURN
      END
C*********************************************************************
      FUNCTION S(A,B,RAB2)
C
C     CALCULATES OVERLAPS FOR UN-NORMALIZED PRIMITIVES
C
C*********************************************************************
      IMPLICIT DOUBLE PRECISION(A-H,O-Z)
      DATA PI/3.1415926535898D0/
      S=(PI/(A+B))**1.5D0*DEXP(-A*B*RAB2/(A+B))
      RETURN
      END
C*********************************************************************
      FUNCTION T(A,B,RAB2)
C
C     CALCULATES KINETIC ENERGY INTEGRALS FOR UN-NORMALIZED PRIMITIVES
C
C*********************************************************************
      IMPLICIT DOUBLE PRECISION(A-H,O-Z)
      DATA PI/3.1415926535898D0/
      T=A*B/(A+B)*(3.0D0-2.0D0*A*B*RAB2/(A+B))*(PI/(A+B))**1.5D0
     $ *DEXP(-A*B*RAB2/(A+B))
      RETURN
      END
C*********************************************************************
      FUNCTION V(A,B,RAB2,RCP2,ZC)
C
C     CALCULATES UN-NORMALIZED NUCLEAR ATTRACTION INTEGRALS
C
C*********************************************************************
      IMPLICIT DOUBLE PRECISION(A-H,O-Z)
      DATA PI/3.1415926535898D0/
      V=2.0D0*PI/(A+B)*F0((A+B)*RCP2)*DEXP(-A*B*RAB2/(A+B))
      V=-V*ZC
      RETURN
      END
C*********************************************************************
      FUNCTION TWOE(A,B,C,D,RAB2,RCD2,RPQ2)
C
C     CALCULATES TWO-ELECTRON INTEGRALS FOR UN-NORMALIZED PRIMITIVES
C     A,B,C,D ARE THE EXPONENTS ALPHA, BETA, ETC.
C     RAB2 EQUALS SQUARED DISTANCE BETWEEN CENTER A AND CENTER B, ETC.
C*********************************************************************
      IMPLICIT DOUBLE PRECISION(A-H,O-Z)
      DATA PI/3.1415926535898D0/
      TWOE=2.0D0*(PI**2.5D0)/((A+B)*(C+D)*DSQRT(A+B+C+D))
     $ *F0((A+B)*(C+D)*RPQ2/(A+B+C+D))
     $ *DEXP(-A*B*RAB2/(A+B)-C*D*RCD2/(C+D))
      RETURN
      END
C*********************************************************************
      SUBROUTINE COLECT(IOP,N,R,ZETA1,ZETA2,ZA,ZB)
C
C     THIS TAKES THE BASIC INTEGRALS FROM COMMON AND ASSEMBLES THE
C     RELEVENT MATRICES, THAT IS S,H,X,XT, AND TWO-ELECTRON INTEGRALS
C
C*********************************************************************
      IMPLICIT DOUBLE PRECISION(A-H,O-Z)
      COMMON/MATRIX/S(2,2),X(2,2),XT(2,2),H(2,2),F(2,2),G(2,2),C(2,2),
     $ FPRIME(2,2),CPRIME(2,2),P(2,2),OLDP(2,2),TT(2,2,2,2),E(2,2)
      COMMON/INT/S12,T11,T12,T22,V11A,V12A,V22A,V11B,V12B,V22B,
     $ V1111,V2111,V2121,V2211,V2221,V2222
C FORM CORE HAMILTONIAN
      H(1,1)=T11+V11A+V11B
      H(1,2)=T12+V12A+V12B
      H(2,1)=H(1,2)
      H(2,2)=T22+V22A+V22B
C FORM OVERLAP MATRIX
      S(1,1)=1.0D0
      S(1,2)=S12
      S(2,1)=S(1,2)
      S(2,2)=1.0D0
C USE CANONICAL ORTHOGONALIZATION
      X(1,1)=1.0D0/DSQRT(2.0D0*(1.0D0+S12))
      X(2,1)=X(1,1)
      X(1,2)=1.0D0/DSQRT(2.0D0*(1.0D0-S12))
      X(2,2)=-X(1,2)
C TRANSPOSE OF TRANSFORMATION MATRIX
      XT(1,1)=X(1,1)
      XT(1,2)=X(2,1)
      XT(2,1)=X(1,2)
      XT(2,2)=X(2,2)
C MATRIX OF TWO-ELE�CTRON INTEGRALS
      TT(1,1,1,1)=V1111
      TT(2,1,1,1)=V2111
      TT(1,2,1,1)=V2111
      TT(1,1,2,1)=V2111
      TT(1,1,1,2)=V2111
      TT(2,1,2,1)=V2121
      TT(1,2,2,1)=V2121
      TT(2,1,1,2)=V2121
      TT(1,2,1,2)=V2121
      TT(2,2,1,1)=V2211
      TT(1,1,2,2)=V2211
      TT(2,2,2,1)=V2221
      TT(2,2,1,2)=V2221
      TT(2,1,2,2)=V2221
      TT(1,2,2,2)=V2221
      TT(2,2,2,2)=V2222
      IF (IOP.EQ.0) GO TO 40
      CALL MATOUT(S,2,2,2,2,4HS   )
      CALL MATOUT(X,2,2,2,2,4HX   )
      CALL MATOUT(H,2,2,2,2,4HH   )
      PRINT 10
   10 FORMAT(//)
      DO 30 I=1,2
      DO 30 J=1,2
      DO 30 K=1,2
      DO 30 L=1,2
      PRINT 20, I,J,K,L,TT(I,J,K,L)
   20 FORMAT(3X,1H(,4I2,2H ),F10.6)
   30 CONTINUE
   40 RETURN
      END
C*********************************************************************
      SUBROUTINE SCF(IOP,N,R,ZETA1,ZETA2,ZA,ZB)
C
C     PERFORMS THE SCF ITERATIONS
C
C*********************************************************************
      IMPLICIT DOUBLE PRECISION(A-H,O-Z)
      COMMON/MATRIX/S(2,2),X(2,2),XT(2,2),H(2,2),F(2,2),G(2,2),C(2,2),
     $ FPRIME(2,2),CPRIME(2,2),P(2,2),OLDP(2,2),TT(2,2,2,2),E(2,2)
      DATA PI/3.1415926535898D0/
C CONVERGENCE CRITERION FOR DENSITY MATRIX
      DATA CRIT/1.0D-4/
C MAXIMUM NUMBER OF ITERATIONS
      DATA MAXIT/25/
C ITERATION NUMBER
      ITER=0
C USE CORE-HAMILTONIAN FOR INITIAL GUESS AT F, I.E. (P=0)
      DO 10 I=1,2
      DO 10 J=1,2
   10 P(I,J)=0.0D0
      IF (IOP.LT.2) GO TO 20
      CALL MATOUT(P,2,2,2,2,4HP   )
C START OF ITERATION LOOP
   20 ITER=ITER+1
      IF (IOP.LT.2) GO TO 40
      PRINT 30, ITER
   30 FORMAT(/,4X,28HSTART OF ITERATION NUMBER = ,I2)
   40 CONTINUE
C FORM TWO-ELECTRON PART OF FOCK MATRIX FROM P
      CALL FORMG
      IF (IOP.LT.2) GO TO 50
      CALL MATOUT(G,2,2,2,2,4HG   )
   50 CONTINUE
C ADD CORE HAMILTONIAN TO GET FOCK MATRIX
      DO 60 I=1,2
      DO 60 J=1,2
      F(I,J) = H(I,J)+G(I,J)
   60 CONTINUE
C CALCULATE ELECTRONIC ENERGY
      EN=0.0D0
      DO 70 I=1,2
      DO 70 J=1,2
      EN=EN+0.5D0*P(I,J)*(H(I,J)+F(I,J))
   70 CONTINUE
      IF (IOP.LT.2) GO TO 90
      CALL MATOUT(F,2,2,2,2,4HF   )
      PRINT 80, EN
   80 FORMAT(///,4X,20HELECTRONIC ENERGY = ,D20.12)
   90 CONTINUE
C TRANSFORM FOCK MATRIX USING G FOR TEMPORARY STORAGE
      CALL MULT(F,X,G,2,2)
      CALL MULT(XT,G,FPRIME,2,2)
C DIAGONALIZE TRANSFORMED FOCK MATRIX
      CALL DIAG(FPRIME,CPRIME,E)
C TRANSFORM EIGENVECTORS TO GET MATRIX C
      CALL MULT(X,CPRIME,C,2,2)
C FORM NEW DENSITY MATRIX
      DO 100 I=1,2
      DO 100 J=1,2
C SAVE PRESENT DENSITY MATRIX
C BEFORE CREATING NEW ONE
      OLDP(I,J)=P(I,J)
      P(I,J)=0.0D0
      DO 100 K=1,1
      P(I,J)=P(I,J)+2.0D0*C(I,K)*C(J,K)
  100 CONTINUE
      IF (IOP.LT.2) GO TO 110
      CALL MATOUT(FPRIME,2,2,2,2,"F'  ")
      CALL MATOUT(CPRIME,2,2,2,2,"C'  ")
      CALL MATOUT(E,2,2,2,2,'E   ')
      CALL MATOUT(C,2,2,2,2,'C   ')
      CALL MATOUT(P,2,2,2,2,'P   ')
  110 CONTINUE
C CALCULATE DELTA
      DELTA=0.0D0
      DO 120 I=1,2
      DO 120 J=1,2
      DELTA=DELTA+(P(I,J)-OLDP(I,J))**2
  120 CONTINUE
      DELTA=DSQRT(DELTA/4.0D0)
      IF (IOP.EQ.0) GO TO 140
      PRINT 130, DELTA
  130 FORMAT(/,4X,39HDELTA(CONVERGENCE OF DENSITY MATRIX) =
     $F10.6,/)
  140 CONTINUE
C CHECK FOR CONVERGENCE
      IF (DELTA.LT.CRIT) GO TO 160
C NOT YET CONVERGED
C TEST FOR MAXIMUM NUMBER OF ITERATIONS
C IF MAXIMUM NUMBER NOT YET REACHED
C GO BACK FOR ANOTHER ITERATION
      IF(ITER.LT.MAXIT) GO TO 20
C SOMETHING WRONG HERE
      PRINT 150
  150 FORMAT(4X,21HNO CONVERGENCE IN SCF)
      STOP
  160 CONTINUE
C CALCULATION CONVERGED IF IT GOT HERE
C ADD NUCLEAR REPULSION TO GET TOTAL ENERGY
      ENT=EN+ZA*ZB/R
      IF (IOP.EQ.0) GO TO 180
      PRINT 170, EN, ENT
  170 FORMAT(//,4X,21HCALCULATION CONVERGED,//,
     $4X,20HELECTRONIC ENERGY = ,D20.12,//,
     $4X,20HTOTAL ENERGY =      ,D20.12   )
  180 CONTINUE
      IF (IOP.NE.1) GO TO 190
C PRINT OUT THE FINAL RESULTS IF
C HAVE NOT DONE SO ALREADY
      CALL MATOUT(G,2,2,2,2,4HG   )
      CALL MATOUT(F,2,2,2,2,4HF   )
      CALL MATOUT(E,2,2,2,2,4HE   )
      CALL MATOUT(C,2,2,2,2,4HC   )
      CALL MATOUT(P,2,2,2,2,4HP   )
  190 CONTINUE
C PS MATRIX HAS MULLIKEN POPULATIONS
      CALL MULT(P,S,OLDP,2,2)
      IF(IOP.EQ.0) GO TO 200
      CALL MATOUT(OLDP,2,2,2,2,4HPS   )
  200 CONTINUE
      RETURN
      END
C*********************************************************************
      SUBROUTINE FORMG
C
C     CALCULATES THE G MATRIX FROM THE DENSITY MATRIX
C     AND TWO-ELECTRON INTEGRALS
C
C*********************************************************************
      IMPLICIT DOUBLE PRECISION(A-H,O-Z)
      COMMON/MATRIX/S(2,2),X(2,2),XT(2,2),H(2,2),F(2,2),G(2,2),C(2,2),
     $FPRIME(2,2),CPRIME(2,2),P(2,2),OLDP(2,2),TT(2,2,2,2),E(2,2)
      DO 10 I=1,2
      DO 10 J=1,2
      G(I,J)=0.0D0
      DO 10 K=1,2
      DO 10 L=1,2
      G(I,J)=G(I,J)+P(K,L)*(TT(I,J,K,L)-0.5D0*TT(I,L,K,J))
   10 CONTINUE
      RETURN
      END
C*********************************************************************
      SUBROUTINE DIAG(F,C,E)
C
C     DIAGONALIZES F TO GIVE EIGENVECTORS IN C AND EIGENVALUES IN E
C     THETA IS THE ANGLE DESCRIBING SOLUTION
C
C*********************************************************************
      IMPLICIT DOUBLE PRECISION(A-H,O-Z)
      DIMENSION F(2,2),C(2,2),E(2,2)
      DATA PI/3.1415926535898D0/
      IF (DABS(F(1,1)-F(2,2)).GT.1.0D-20) GO TO 10
C HERE IS SYMMETRY DETERMINED SOLUTION (HOMONUCLEAR DIATOMIC)
      THETA=PI/4.0D0
      GO TO 20
   10 CONTINUE
C SOLUTION FOR HETERONUCLEAR DIATOMIC
      THETA=0.5D0*DATAN(2.0D0*F(1,2)/(F(1,1)-F(2,2)))
   20 CONTINUE
      C(1,1)=DCOS(THETA)
      C(2,1)=DSIN(THETA)
      C(1,2)=DSIN(THETA)
      C(2,2)=-DCOS(THETA)
      E(1,1)=F(1,1)*DCOS(THETA)**2+F(2,2)*DSIN(THETA)**2
     $ +F(1,2)*DSIN(2.0D0*THETA)
      E(2,2)=F(2,2)*DCOS(THETA)**2+F(1,1)*DSIN(THETA)**2
     $ -F(1,2)*DSIN(2.0D0*THETA)
      E(2,1)=0.0D0
      E(1,2)=0.0D0
C ORDER EIGENVALUES AND EIGENVECTORS
      IF (E(2,2).GT.E(1,1)) GO TO 30
      TEMP=E(2,2)
      E(2,2)=E(1,1)
      E(1,1)=TEMP
      TEMP=C(1,2)
      C(1,2)=C(1,1)
      C(1,1)=TEMP
      TEMP=C(2,2)
      C(2,2)=C(2,1)
      C(2,1)=TEMP
   30 RETURN
      END
C*********************************************************************
      SUBROUTINE MULT(A,B,C,IM,M)
C
C     MULTIPLIES TWO SQUARE MATRICES A AND B TO GET C
C
C*********************************************************************
      IMPLICIT DOUBLE PRECISION(A-H,O-Z)
      DIMENSION A(IM,IM),B(IM,IM),C(IM,IM)
      DO 10 I=1,M
      DO 10 J=1,M
      C(I,J)=0.0D0
      DO 10 K=1,M
   10 C(I,J)=C(I,J)+A(I,K)*B(K,J)
      RETURN
      END
C*********************************************************************
      SUBROUTINE MATOUT(A,IM,IN,M,N,LABEL)
C
C     PRINT MATRICES OF SIZE M BY N
C
C*********************************************************************
      IMPLICIT DOUBLE PRECISION(A-H,O-Z)
      DIMENSION A(IM,IN)
      IHIGH=0
   10 LOW=IHIGH+1
      IHIGH=IHIGH+5
      IHIGH=MIN(IHIGH,N)
      PRINT 20, LABEL,(I,I=LOW,IHIGH)
   20 FORMAT(///,3X,5H THE ,A4,6H ARRAY,/,15X,5(10X,I3,6X)//)
      DO 30 I=1,M
   30 PRINT 40, I,(A(I,J),J=LOW,IHIGH)
   40 FORMAT(I10,5X,5(1X,D18.10))
      IF (N-IHIGH) 50,50,10
   50 RETURN
      END
         
\end{verbatim}

\newpage
\begin{verbatim}
    STO-3G FOR ATOMIC NUMBERS  2.00 AND  1.00


R          ZETA1      ZETA2      S12        T11

1.463200   2.092500   1.240000   0.450770   2.164313


T12        T22        V11A       V12A       V22A

0.167013   0.760033  -4.139827  -1.102912  -1.265246


V11B       V12B       V22B       V1111      V2111

-0.677230  -0.411305  -1.226615   1.307152   0.437279


V2121      V2211      V2221      V2222

0.177267   0.605703   0.311795   0.774608



 THE S    ARRAY
                        1                  2
      1        0.1000000000E+01   0.4507704116E+00
      2        0.4507704116E+00   0.1000000000E+01



 THE X    ARRAY
                        1                  2
      1        0.5870642812E+00   0.9541310722E+00
      2        0.5870642812E+00  -0.9541310722E+00



 THE H    ARRAY
                        1                  2
      1       -0.2652744703E+01  -0.1347205024E+01
      2       -0.1347205024E+01  -0.1731828436E+01



( 1 1 1 1 )  1.307152
( 1 1 1 2 )  0.437279
( 1 1 2 1 )  0.437279
( 1 1 2 2 )  0.605703
( 1 2 1 1 )  0.437279
( 1 2 1 2 )  0.177267
( 1 2 2 1 )  0.177267
( 1 2 2 2 )  0.311795
( 2 1 1 1 )  0.437279
( 2 1 1 2 )  0.177267
( 2 1 2 1 )  0.177267
( 2 1 2 2 )  0.311795
( 2 2 1 1 )  0.605703
( 2 2 1 2 )  0.311795
( 2 2 2 1 )  0.311795
( 2 2 2 2 )  0.774608



 THE P    ARRAY
                        1                  2
      1        0.0000000000E+00   0.0000000000E+00
      2        0.0000000000E+00   0.0000000000E+00

 START OF ITERATION NUMBER =  1



 THE G    ARRAY
                        1                  2
      1        0.0000000000E+00   0.0000000000E+00
      2        0.0000000000E+00   0.0000000000E+00



 THE F    ARRAY
                        1                  2
      1       -0.2652744703E+01  -0.1347205024E+01
      2       -0.1347205024E+01  -0.1731828436E+01



 ELECTRONIC ENERGY =   0.000000000000E+00



 THE F'   ARRAY
                        1                  2
      1       -0.2439732411E+01  -0.5158386047E+00
      2       -0.5158386047E+00  -0.1538667186E+01



 THE C'   ARRAY
                        1                  2
      1        0.9104452570E+00   0.4136295856E+00
      2        0.4136295856E+00  -0.9104452570E+00



 THE E    ARRAY
                        1                  2
      1       -0.2674085994E+01   0.0000000000E+00
      2        0.0000000000E+00  -0.1304313603E+01



 THE C    ARRAY
                        1                  2
      1        0.9291467304E+00  -0.6258569539E+00
      2        0.1398330503E+00   0.1111511265E+01



 THE P    ARRAY
                        1                  2
      1        0.1726627293E+01   0.2598508430E+00
      2        0.2598508430E+00   0.3910656393E-01

 DELTA(CONVERGENCE OF DENSITY MATRIX) =   0.882867


 START OF ITERATION NUMBER =  2



 THE G    ARRAY
                        1                  2
      1        0.1262330044E+01   0.3740040563E+00
      2        0.3740040563E+00   0.9889530699E+00



 THE F    ARRAY
                        1                  2
      1       -0.1390414659E+01  -0.9732009679E+00
      2       -0.9732009679E+00  -0.7428753661E+00



 ELECTRONIC ENERGY =  -0.414186268681E+01



 THE F'   ARRAY
                        1                  2
      1       -0.1406043275E+01  -0.3627102456E+00
      2       -0.3627102456E+00  -0.1701365815E+00



 THE C'   ARRAY
                        1                  2
      1        0.9649913726E+00   0.2622816249E+00
      2        0.2622816249E+00  -0.9649913726E+00



 THE E    ARRAY
                        1                  2
      1       -0.1504626781E+01   0.0000000000E+00
      2        0.0000000000E+00  -0.7155307568E-01



 THE C    ARRAY
                        1                  2
      1        0.8167630145E+00  -0.7667520795E+00
      2        0.3162609186E+00   0.1074704427E+01



 THE P    ARRAY
                        1                  2
      1        0.1334203644E+01   0.5166204425E+00
      2        0.5166204425E+00   0.2000419373E+00

 DELTA(CONVERGENCE OF DENSITY MATRIX) =   0.279176


 START OF ITERATION NUMBER =  3



 THE G    ARRAY
                        1                  2
      1        0.1201346300E+01   0.3038061741E+00
      2        0.3038061741E+00   0.9284329600E+00



 THE F    ARRAY
                        1                  2
      1       -0.1451398403E+01  -0.1043398850E+01
      2       -0.1043398850E+01  -0.8033954759E+00



 ELECTRONIC ENERGY =  -0.422649172562E+01



 THE F'   ARRAY
                        1                  2
      1       -0.1496305530E+01  -0.3629699437E+00
      2       -0.3629699437E+00  -0.1529380263E+00



 THE C'   ARRAY
                        1                  2
      1        0.9694747516E+00   0.2451911622E+00
      2        0.2451911622E+00  -0.9694747516E+00



 THE E    ARRAY
                        1                  2
      1       -0.1588104746E+01   0.0000000000E+00
      2        0.0000000000E+00  -0.6113881008E-01



 THE C    ARRAY
                        1                  2
      1        0.8030885047E+00  -0.7810630108E+00
      2        0.3351994916E+00   0.1068948958E+01



 THE P    ARRAY
                        1                  2
      1        0.1289902293E+01   0.5383897171E+00
      2        0.5383897171E+00   0.2247173984E+00

 DELTA(CONVERGENCE OF DENSITY MATRIX) =   0.029662


 START OF ITERATION NUMBER =  4



 THE G    ARRAY
                        1                  2
      1        0.1194670199E+01   0.2971625826E+00
      2        0.2971625826E+00   0.9218705199E+00



 THE F    ARRAY
                        1                  2
      1       -0.1458074504E+01  -0.1050042442E+01
      2       -0.1050042442E+01  -0.8099579160E+00



 ELECTRONIC ENERGY =  -0.422752275334E+01



 THE F'   ARRAY
                        1                  2
      1       -0.1505447474E+01  -0.3630336096E+00
      2       -0.3630336096E+00  -0.1528937446E+00



 THE C'   ARRAY
                        1                  2
      1        0.9698136474E+00   0.2438472663E+00
      2        0.2438472663E+00  -0.9698136474E+00



 THE E    ARRAY
                        1                  2
      1       -0.1596727643E+01   0.0000000000E+00
      2        0.0000000000E+00  -0.6161357601E-01



 THE C    ARRAY
                        1                  2
      1        0.8020052055E+00  -0.7821753152E+00
      2        0.3366806982E+00   0.1068483355E+01



 THE P    ARRAY
                        1                  2
      1        0.1286424699E+01   0.5400393450E+00
      2        0.5400393450E+00   0.2267077850E+00

 DELTA(CONVERGENCE OF DENSITY MATRIX) =   0.002318


 START OF ITERATION NUMBER =  5



 THE G    ARRAY
                        1                  2
      1        0.1194147845E+01   0.2966515832E+00
      2        0.2966515832E+00   0.9213575914E+00



 THE F    ARRAY
                        1                  2
      1       -0.1458596858E+01  -0.1050553441E+01
      2       -0.1050553441E+01  -0.8104708445E+00



 ELECTRONIC ENERGY =  -0.422752909612E+01



 THE F'   ARRAY
                        1                  2
      1       -0.1506156505E+01  -0.3630388891E+00
      2       -0.3630388891E+00  -0.1529058377E+00



 THE C'   ARRAY
                        1                  2
      1        0.9698390734E+00   0.2437461212E+00
      2        0.2437461212E+00  -0.9698390734E+00



 THE E    ARRAY
                        1                  2
      1       -0.1597397746E+01   0.0000000000E+00
      2        0.0000000000E+00  -0.6166459619E-01



 THE C    ARRAY
                        1                  2
      1        0.8019236265E+00  -0.7822589536E+00
      2        0.3367921305E+00   0.1068448236E+01



 THE P    ARRAY
                        1                  2
      1        0.1286163006E+01   0.5401631334E+00
      2        0.5401631334E+00   0.2268578784E+00

 DELTA(CONVERGENCE OF DENSITY MATRIX) =   0.000174


 START OF ITERATION NUMBER =  6



 THE G    ARRAY
                        1                  2
      1        0.1194108547E+01   0.2966131916E+00
      2        0.2966131916E+00   0.9213190058E+00



 THE F    ARRAY
                        1                  2
      1       -0.1458636156E+01  -0.1050591833E+01
      2       -0.1050591833E+01  -0.8105094301E+00



 ELECTRONIC ENERGY =  -0.422752913203E+01



 THE F'   ARRAY
                        1                  2
      1       -0.1506209810E+01  -0.3630392881E+00
      2       -0.3630392881E+00  -0.1529068392E+00



 THE C'   ARRAY
                        1                  2
      1        0.9698409800E+00   0.2437385353E+00
      2        0.2437385353E+00  -0.9698409800E+00



 THE E    ARRAY
                        1                  2
      1       -0.1597448132E+01   0.0000000000E+00
      2        0.0000000000E+00  -0.6166851652E-01



 THE C    ARRAY
                        1                  2
      1        0.8019175078E+00  -0.7822652261E+00
      2        0.3368004878E+00   0.1068445602E+01



 THE P    ARRAY
                        1                  2
      1        0.1286143379E+01   0.5401724156E+00
      2        0.5401724156E+00   0.2268691372E+00

 DELTA(CONVERGENCE OF DENSITY MATRIX) =   0.000013



 CALCULATION CONVERGED

 ELECTRONIC ENERGY =  -0.422752913203E+01

 TOTAL ENERGY =       -0.286066199152E+01



 THE PS   ARRAY
                        1                  2
      1        0.1529637121E+01   0.1119927796E+01
      2        0.6424383099E+00   0.4703628793E+00       
\end{verbatim}
  
\newpage
\theendnotes
\addcontentsline{toc}{section}{注释}
